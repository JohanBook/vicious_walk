\documentclass[a4paper,12pt]{article}
\usepackage{a4wide}
\usepackage[utf8]{inputenc}
\bibliographystyle{natbib}
\usepackage{graphicx}
\usepackage{amsmath}
\usepackage{subcaption, caption}
\usepackage{float}
\usepackage{tikz}
\usetikzlibrary{shapes.geometric, arrows}
\usepackage{booktabs}

% Commands
\newcommand{\D} {\mathrm d} %
\newcommand{\df}[2]{\frac {\mathrm d #1} {\mathrm d #2}}
\newcommand{\ddf}[2]{\frac {\mathrm d^2 #1} {\mathrm d #2^2}}
\newcommand{\pf}[2]{\frac {\partial #1} {\partial #2}}
\newcommand{\ppf}[2]{\frac {\partial^2 #1} {\partial #2^2}}

% Common Sets
\newcommand{\N}{\mathbb{N}}
\newcommand{\Z}{\mathbb{Q}}
\newcommand{\Q}{\mathbb{Q}}
\newcommand{\R}{\mathbb{R}}

% Java stuff
\usepackage{listings}
\usepackage{color}

\definecolor{dkgreen}{rgb}{0,0.6,0}
\definecolor{gray}{rgb}{0.5,0.5,0.5}
\definecolor{mauve}{rgb}{0.58,0,0.82}

\lstset{frame=tb,
  language=Java,
  aboveskip=3mm,
  belowskip=3mm,
  showstringspaces=false,
  columns=flexible,
  basicstyle={\small\ttfamily},
  numbers=none,
  numberstyle=\tiny\color{gray},
  keywordstyle=\color{blue},
  commentstyle=\color{dkgreen},
  stringstyle=\color{mauve},
  breaklines=true,
  breakatwhitespace=true,
  tabsize=3
}

\begin{document}

%%%%%%%%%%%%%%%%%%%%%%%%%%%%%%%%%%%%%%%%%%%%%%%%%%%%%%%%%%%%%%%%%%%%%%%%%%
% Title page
%%%%%%%%%%%%%%%%%%%%%%%%%%%%%%%%%%%%%%%%%%%%%%%%%%%%%%%%%%%%%%%%%%%%%%%%%%%
\begin{titlepage}
% flushright puts it towards the right of the page
\begin{flushright}
% our preprint number yy-nn (year-number), ask your supervisor how to get this
% some indication of the date
13 Sep 2015\\
\end{flushright}
\vfill
\begin{center}
% put in line breaks to make the title look nicer and provide more
% space between the lines
{\large\bf MODELLING ROLE-PLAYING CHARACTER CLASS PERFORMANCES USING RANDOM WALKS\\[3mm] 
FYTN03}
\\[3cm]
{\bf Johan Book}
\\[5mm]
{Department of Astronomy and Theoretical Physics, Lund University}
\\[2cm]
{Project done together with Ola Olsson}
\vfill
% use the eps version for latex and the pdf version for pdflatex 
%\includegraphics[height=4cm]{logocLUeng.eps}
\includegraphics[height=4cm]{Lunds_universitet_seal.png}
\end{center}
\end{titlepage}
%%%%%%%%%%%%%%%%%%%%%%%%%%%%%%%%%%%%%%%%%%%%%%%%%%%%%%%%%%%%%%%%%%%%%%%%%%%
% a page with the abstract and popular description
\thispagestyle{empty} % do not count pages just yet

\vfill

\vfill

\newpage

\tableofcontents

%==================== Introduction ====================
\newpage
\section{Introduction} \label{sec:Introduction}
The performance of each class in role-playing games such as DnD and Pathfinder is widely discussed.
The aim of this study is the investigate said performance for individual characters of different classes picking fights at random. 
It will build upon a random-walk model where combatants must fight to until death.
The model will serve measure of solo performance.

%==================== Theory ====================
\newpage
\section{Theory}\label{sec:Theory}



\subsection{General role-playing and Pathfinder statistics}
A common role-playing game is Pathfinder (which is a variant of DnD).
The player gets to create a character of a certain class to fight enemies. Each character has some hit-points which corresponds to a score determining how healthy or hurt the character is. In order to kill an enemy the character must inflict damage on the enemy reducing the enemy's hit-points. After the character has slain an amount of enemies it levels up gaining more hit-points, a higher attack damage and is allowed to use more advanced abilities.
\\
\\
The character has a set of ability scores which correspond to e.g. how strong, intelligent or charismatic the character is. This does in turn affect how well a character can perform any kind of action such as damaging an enemy, climb a tree or casting a spell.

\begin{table}[H] 
	\centering
	\caption{Generalized ability scores for some classes. \label{table:Ability_scores}}
	\begin{tabular}{@{}lllllll@{}}
	\toprule
	Class & Strength & Dexterity & Constitution & Wisdom & Intelligence & Charisma \\ \midrule
 	Paladin & 18 	& 11 	& 13 	& 8 	& 8		& 14 \\
 	Fighter & 18 	& 14 	& 13 	& 8 	& 11 	& 8 \\
 	Rogue 	& 13	& 18 	& 11 	& 8 	& 14 	& 8 \\
 	Ranger 	& 14 	& 18 	& 11 	& 13 	& 8 	& 8 \\
 	Wizard 	& 8 	& 14 	& 13 	& 8 	& 18 	& 11 \\
 	Cleric 	& 13 	& 8 	& 14 	& 18 	& 8 	& 11 \\
 	Skeleton& 14 	& 14 	& - 	& 8 	& - 	& 10 \\
 	Orc 	& 16 	& 14 	& 13 	& 8 	& 11 	& 8 \\ \bottomrule
	\end{tabular}
\end{table}

\noindent
Further explanation of the statistics can be found in the Pathfinder core rule book \cite{rule_book}.

 
\newpage
\noindent
Each class has some fixed attributes (or which here is considered fixed), such as hit-points gained when levelling.
The classes will have some armour which is presented with an armour class score.

\begin{table}[H] 
	\centering
	\caption{Statistics for different Pathfinder classes. \label{table:Stats}}
	\begin{tabular}{@{}lllll@{}}
	\toprule
	Class & Speed [feet] & Base hit-points & Hit-points per level & Armour class \\ \midrule
 	Paladin & 20 	& 12 	& 5 	& 23 	\\
 	Fighter & 30 	& 12 	& 5 	& 20 	\\
 	Rogue 	& 25	& 8 	& 4 	& 17 	\\
 	Ranger 	& 35 	& 10 	& 5 	& 20 	\\
 	Wizard 	& 30 	& 6 	& 3 	& 16 	\\
 	Cleric 	& 30 	& 10 	& 4 	& 23 	\\
 	Skeleton& 30 	& 6 	& 3 	& 16 	\\
 	Orc 	& 20 	& 12 	& 5 	& 18 	\\ \bottomrule
	\end{tabular}
\end{table}

\noindent
Most of the presented classes does also have a special attribute. Below some of them are displayed, somewhat modified and simplified.
First strike does in this context designate the first strike a character does on another, initiating a combat.

\begin{table}[H] 
	\centering
	\caption{Some bonuses and whether classes is ranged or not.\label{table:Bonuses}}
	\begin{tabular}{@{}lll@{}}
	\toprule
	Class & Is ranged & Bonus \\ \midrule
 	Paladin & False 	& Can heal during combat. Extra damage against skeleton and orc.	\\
 	Fighter & False 	& \\
 	Rogue 	& False*	& Damage bonus if first strike (does not affect skeleton). 	\\
 	Ranger 	& True 		& \\
 	Wizard 	& True 		& Can cast spells.\\
 	Cleric 	& False 	& Extra damage against skeleton. Regain full hp after combat. \\
 		& & Can cast spells.\\
 	Skeleton& False 	& \\
 	Orc 	& False 	& \\ \bottomrule
	\end{tabular}
	\caption*{* Can be considered to be ranged.}
\end{table}

%==================== Methods ====================
\newpage
\section{Methods}\label{sec:Methods}
\subsection{Basic idea}
Characters of different classes will be positioned in some space $V$. They are then let to wander randomly until two meet and initiate a fight until death. This is meant to correspond to how often these characters meet and fight in some fictional world. 
In most game-plays characters will form teams. However this will be neglected for the sake of simplicity.

\subsection{Initial character placement}
Let the number of characters to be placed from class $i$ be $C_i$.
Define $M$ as the number of discrete possible positions in $V$.
For each allowed position do the following:
\begin{enumerate}
	\item Draw a random number $X$. If $X \leq C_i/M$ place a character of class $i$ at the probed position and decrease $C_i$ by 1.
		Do only place one character at the same position.
	\item Decrease $M$ by 1.
\end{enumerate}


\subsection{Movement and basic combat structure}
Combats have been greatly simplified. Movement and combat was done by an algorithm presented below.
Damage was calculated according to the rules of combat specified in the Pathfinder core rule book \cite{rule_book}.
\begin{enumerate}
	\item Pick one alive character $P$ at random.
	\item Move $P$ using a random step with a maximum step length of the speed of the character depicted in table \ref{table:Stats}.
	\item Pick any other character $O$ in a range less than the character speed. If none can be found skip remaining steps.
	\item Let $P$ inflict damage on $O$.
	\item If $O$ is not alive increase the level of $P$ by 1 and skip remaining. 
	If $P$ is ranged and stroke a first strike go to 4 else let $O$ attack $P$.
	\item If $P$ is alive go to step 4 else increase the level of $O$ by 1.
\end{enumerate}
One step consist of the process above being run $N$ times where $N$ is the current number of alive characters.
After one step is completed all characters are healed and their spells are reset.


\subsection{A continuous model}
Instead of the discrete model proposed in \cite{lecture_notes} a continuous one was used.
The reason for this choice is that for the studied types of reactions do a continuous model seem more realistic.
One major drawback is that one must at each time step check a particles distance from every other particle - which is a resource-intensive process.

\subsection{Comments}
The speeds presented in table \ref{table:Stats} was reduced by $1/7$ in order to slow down the simulation.


%==================== Results ====================
\newpage
\section{Results}\label{sec:Results}
\begin{figure}[h]
	\centering
	\includegraphics[width=12cm]{evol2}
	\caption{Evolution of the system. \label{fig:Evolution}}
\end{figure}

\noindent
The average number of characters in each class was studied during 1000 simulations.
Note that in the graph only a few data points are shown by points in order to increase readability.

\begin{table}[H] 
	\parbox{.45\linewidth}{
	\centering
	\caption{The parameters used to generate the data.\label{table:Simulation_Settings}}
	\begin{tabular}{@{}ll@{}}
		\toprule
		Parameter & Value  \\ \midrule
		Simulations & 1000 \\
 		Steps & 1000 		\\
 		System size & $1000^2$ 	 \\
 		Characters of each class 	& 200 	\\ \bottomrule
	\end{tabular}
	}
	\hfill
	\parbox{.45\linewidth}{
	\centering
	\caption{The percetage surviors of each class.\label{table:Final}}
	\begin{tabular}{@{}ll@{}}
		\toprule
		Class 		& Percentage survivors  \\ \midrule
 		Paladin 	& 38.8 \%\\
 		Fighter 	& 25.6 \%\\
 		Rogue 		& 3.9 \%\\
 		Ranger 		& 9.9 \%\\
 		Wizard 		& 2.6 \%\\
 		Cleric 		& 3.8 \%\\
 		Skeleton 	& 0.8 \%\\
 		Orc 		& 14.6 \% 	\\ \bottomrule
	\end{tabular}
}
\end{table}

%==================== Analysis ====================
\newpage
\section{Analysis}\label{sec:Analysis}
In fig. \ref{fig:Evolution} does the paladin character class dominate - which agrees with the general consensus.
The fighters and orcs did perform well which is not very surprising considering they are specialized in close combat.
The only close combat class which did not perform well was skeletons. This is due to that this class is naturally weak.
\\
\\
The result seem realistic considering the model set-up.
It implies that solo combats favours classes specialized in close combat.
The remaining classes are as most effective when working as a team.
\\
\\
One can draw the conclusion that when constructing a team of characters at least one need to be specialized in close combat.
From this simulation can the paladin be considered the optimal choice.


%%%%%%%%%%%%%%%%%%%%%%%%%%%%%%%%%%%%%%%%%%%%%%%%%%%%%%%%%%%%%%%%%%%%%%%%%%
% Appendix
%%%%%%%%%%%%%%%%%%%%%%%%%%%%%%%%%%%%%%%%%%%%%%%%%%%%%%%%%%%%%%%%%%%%%%%%%%

\appendix
\newpage
\section{Code: Main}
\begin{lstlisting}
package src;

import gui.GUI;
import java.util.ArrayList;
import java.util.Map;
import character.Character;
import character.Class;
import react.Collision;
import settings.Settings;
import util.Pair;
import util.Printer;
import util.Random;
import util.Util;

public class Main {
	private final ArrayList<Character> characters;
	private final Map<Class, Double> densities;
	private final Settings settings;

	public Main() {
		settings = new Settings();
		Initializer placer = new Initializer(settings);
		characters = placer.getInitalCharacters();

		// Assumes equal amount of each class
		densities = Util.initiateMap(settings.numbPALADIN);
	}

	// Simulates all time-steps. If supplied with a printer writes data to file
	// Stores densities of each step in map
	public void simulate(Printer printer, Map<Class, Double>[] map) {
		// Print file header
		if (printer != null)
			printer.write("#" + Class.classesToString());

		// Take each step
		for (int step = 0; step < Settings.steps; step++) {
			int num = characters.size();
			for (int j = 0; j < num; j++)
				simulate();

			// Heal characters
			healAllCharacters();

			if (printer != null)
				printer.write(step + "\t" + densitiesToString(false));

			if (map != null)
				map[step] = Util.addMaps(map[step], Util.copyMap(densities));
		}
		if (printer != null)
			printer.close();
	}

	// Simulates time-steps together with a GUI
	public GUI createGUI() {
		return new GUI(characters, settings.getDimensions());
	}

	// Put all densities into a string
	public String densitiesToString(boolean normalize) {
		String string = "";
		int norm = normalize ? densities.size() : 1;
		for (Map.Entry<?, Double> entry : densities.entrySet())
			string += entry.getValue() / norm + "\t";
		return string;
	}

	// Simulates one time-step
	private void simulate() {
		// Create lists to store Characters to remove
		ArrayList<Character> remove_list = new ArrayList<Character>();

		// Pick and move one random Character
		Character selected_Character = characters.get(Random.nextInt(characters
				.size()));
		selected_Character.move();

		// check for reactions/fights
		for (int i = 0; i < characters.size(); i++) {
			if (characters.get(i) == selected_Character)
				continue;

			// Check for reaction
			Pair<Boolean, Character> reaction = Collision.check(
					selected_Character, characters.get(i));

			// if there is no reaction continue
			if (!reaction.first)
				continue;

			// If there is a fight remove the one who loses
			if (reaction.second != null) {
				remove_list.add(reaction.second);
				densities.compute(reaction.second.getType(), (k, v) -> v - 1);
			}
			break;
		}

		// Remove and add Character
		for (Character Character : remove_list)
			characters.remove(Character);
	}

	private void healAllCharacters() {
		for (Character character : characters)
			character.heal();
	}

	// Main [int: number of simulations][boolean: display GUI]
	@SuppressWarnings("unchecked")
	public static void main(String[] args) {
		int numbSim = 1; // number of simulations to run
		boolean useGUI = true; // whether to display GUI
		if (args.length >= 1) {
			numbSim = Integer.parseInt(args[0]);
			useGUI = false;
		}
		if (args.length >= 2)
			useGUI = true;

		// Initiate a printer to write the final result
		// from each simulation and a map to record the average.
		Printer printer_summary = new Printer("summary");
		printer_summary.write("#Sim\t" + Class.classesToString());

		// Setup map for recording final values and average for each step
		Map<Class, Double>[] step_average = new Map[Settings.steps];
		for (int i = 0; i < Settings.steps; i++)
			step_average[i] = Util.initiateMap(0);

		for (int i = 0; i < numbSim; i++) {
			Main simulation = new Main();

			// Initiate GUI if used
			GUI gui = null;
			if (useGUI)
				gui = simulation.createGUI();

			// Run a simulation and print data to file
			simulation.simulate(new Printer("result_sim_" + (i + 1)),
					step_average);
			printer_summary
					.write(i + "\t" + simulation.densitiesToString(true));

			System.out.println("Simulation " + (i + 1) + "/" + (numbSim)
					+ " completed");

			// Close GUI
			if (useGUI)
				gui.dispose();
		}
		printer_summary.close();

		// Calculate average for each step and print
		Printer average_each_step = new Printer("step_average");
		average_each_step.write("Step\t" + Class.classesToString());
		for (int step = 0; step < Settings.steps; step++) {
			String[] string = Util.mapToString(step_average[step], numbSim,
					false);
			average_each_step.write(step + "\t" + string[1]);
		}
		average_each_step.close();

		// Calculate final average and put into string
		String[] results = util.Util.mapToString(
				step_average[Settings.steps - 1], numbSim, true);

		// Print the final average from all simulations
		printer_summary = new Printer("summary_average");
		printer_summary.write("#" + results[0]);
		printer_summary.write(results[1]);
		printer_summary.close();

		System.out.println("All simulations complete");
		System.exit(0);
	}
}
\end{lstlisting}

\newpage
\section{Code: Initializer}
\begin{lstlisting}
package src;

import java.util.ArrayList;
import java.util.Random;

import character.Character;
import character.Class;
import settings.Settings;

public class Initializer {
	private Random random = new Random();

	private int numbPALADIN;
	private int numbFIGHTER;
	private int numbROGUE;
	private int numbRANGER;
	private int numbWIZARD;
	private int numbCLERIC;
	private int numbSKELETON;
	private int numbORC;

	private double M2;

	private final Settings settings;

	public Initializer(Settings settings) {
		this.settings = settings;
		numbPALADIN = settings.numbPALADIN;
		numbFIGHTER = settings.numbFIGHTER;
		numbROGUE = settings.numbROGUE;
		numbRANGER = settings.numbRANGER;
		numbWIZARD = settings.numbWIZARD;
		numbCLERIC = settings.numbCLERIC;
		numbSKELETON = settings.numbSKELETON;
		numbORC = settings.numbORC;

		M2 = Math.pow(settings.M, settings.n);
	}

	public ArrayList<Character> getInitalCharacters() {
		ArrayList<Character> molecules = new ArrayList<Character>();

		// Place molecules evenly
		for (int x = 0; x < settings.M; x++) 
			for (int y = 0; y < settings.M; y++) {
				double[] position = new double[settings.n];
				position[0] = (int) settings.pixel / 2 + x * settings.pixel;
				position[1] = (int) settings.pixel / 2 + y * settings.pixel;
				chanceOfCharacterPlacement_Evenly(molecules, position);
			}
		return molecules;
	}

	// For evenly placed molecules
	private void chanceOfCharacterPlacement_Evenly(
			ArrayList<Character> molecules, double[] position) {

		// Create molecule if random nr [0,1] < nr small_cubes/nrMolecules

		if (random.nextDouble() <= numbPALADIN / M2 && numbPALADIN > 0) {
			molecules.add(new Character(Class.PALADIN, position));
			numbPALADIN--;
		}

		else if (random.nextDouble() <= numbFIGHTER / M2 && numbFIGHTER > 0) {
			molecules.add(new Character(Class.FIGHTER, position));
			numbFIGHTER--;
		}

		else if (random.nextDouble() <= numbROGUE / M2 && numbROGUE > 0) {
			molecules.add(new Character(Class.ROGUE, position));
			numbROGUE--;
		}

		else if (random.nextDouble() <= numbRANGER / M2 && numbRANGER > 0) {
			molecules.add(new Character(Class.RANGER, position));
			numbRANGER--;
		}

		else if (random.nextDouble() <= numbWIZARD / M2 && numbWIZARD > 0) {
			molecules.add(new Character(Class.WIZARD, position));
			numbWIZARD--;
		}

		else if (random.nextDouble() <= numbCLERIC / M2 && numbCLERIC > 0) {
			molecules.add(new Character(Class.CLERIC, position));
			numbCLERIC--;
		}

		else if (random.nextDouble() <= numbSKELETON / M2 && numbSKELETON > 0) {
			molecules.add(new Character(Class.SKELETON, position));
			numbSKELETON--;
		}

		else if (random.nextDouble() <= numbORC / M2 && numbORC > 0) {
			molecules.add(new Character(Class.ORC, position));
			numbORC--;
		}

		M2--;
	}
}
\end{lstlisting}

%%%%%%%%%%%%%%%%%%%%%%%%%%%%%%%%%%%%%%%%%%%%%%%%%%%%%%%%%%%%%%%%%%%%%%%%%%
% Bibliography
%%%%%%%%%%%%%%%%%%%%%%%%%%%%%%%%%%%%%%%%%%%%%%%%%%%%%%%%%%%%%%%%%%%%%%%%%%
\begin{thebibliography}{99}
\bibitem{lecture_notes}
  Lund University\\
  \emph{SP, Lecture 2, FYTN03 Computational Physics}.
  
\bibitem{rule_book}
  Jason Bulmahn, Gary Gygax, Dave Arneson - Paizo \\
  \emph{Pathfinder roleplaying game: core rulebook}.\\
  Paizo Publishing 2009.

\end{thebibliography}

\end{document}
